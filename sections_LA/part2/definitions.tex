\section{Definitions} \label{sec:sed-ultrices}

\subsection{Axioms that make a vector space}

A vector space is always defined over a field. Fields, in simple terms, can be thought of as number systems. It can be $\mathbb{R}$ or the field of the real numbers; it can also be $\mathbb{C}$, which is the field of the complex numbers. 

There are two axioms that are central to the formation of vector spaces. Vector space is defined to be a space and vectors are said to be its elements such that:

\begin{enumerate}
\item{If two vectors from this space are added, then the resulting vector is also an element of this space. i.e. $v,w \in \mathbb{V} \implies v+w \in \mathbb{V}$}
\item{If we multiply a vector from this space with a scalar, which is an element of the field, then the resulting quantity or vector should be an element of the vector space.}
\end{enumerate}

\subsection{More axioms}

Based on these fundamental axioms, we can assert more axioms that build up the notion of a vector space. All properties of the vector space can be derived from these axioms.


\begin{enumerate}
    \item \textbf{Commutativity of addition} v+w=w+v
    \item \textbf{Existence of zero vector} 
    v+0=v, and it is unique
    \item \textbf{Associativity of addition}
    v+(w+y)=(v+w)+y
    \item \textbf{Existence of additive inverses}
    v+(-v)=0
    \item \textbf{Identity multiplication}
    1.v=v
    \item \textbf{Associalivity of Scalar multiplication} a.(bv)=ab.v
    \item \textbf{Distributivity, both over scalars and vectors} (a+b).v = a.v +b.v 
    and a.(v+w)=a.v+a.w
\end{enumerate}

\subsubsection{Examples}
Here, we take some examples and discuss whether these are vector spaces or not. 

\begin{enumerate}
    \item The set of continuous functions from $[0, 1]$ to $\mathbb{R}$ (or $\mathbb{C}$) under the usual addition and scalar multiplication operations.
    \item Polynomials of degree $\leq n$ with coefficients from a field $F$. (Polynomials of degree exactly $n$ do NOT form a vector space.)
    \item Polynomials with integer coefficients do NOT form a vector space.
    \item $m \times n$ matrices with complex/real entries.
    \item The set of all differentiable functions $x, y : \mathbb{R} \to \mathbb{R}$ satisfying $x' = 2x + 3y$, $y' = 4x + 5y$ forms a vector space over $\mathbb{R}$.
\end{enumerate}

\subsection{Subspaces}
A subspace is a subset of a vector space that is a vector space in its own right, as it satisfies the fundamental two axioms of the vector spaces, closure under addition and scalar multiplication. 


\subsubsection{Examples}
It is always possible to rigorously investigate a subspace for its properties, such as:

\begin{enumerate}
    \item If the set of all the differentiable functions form a vector space, then the functions that satisfy a certain differential equation would follow a subspace. 
    \item If we consider the vector space spanned by all the matrices with real entries then upper triangular matrices form a subspace. 
    \item If we consider the vector space created by nx1 column vectors, then are the vectors having only positive entries a subspace of this space? No! It is not closed under scalar multiplication. 
\end{enumerate}

\textbf{Important Definition}

\begin{outline}
    If there is a set of vectors called $S$, then the vector that is created by all possible linear combinations of these vectors is called the span of the set $S$; it is denoted by the notation $L(S)$.
\end{outline}