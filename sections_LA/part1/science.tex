\section{Science} \label{sec:Science}

\subsection{Computer Science}
Computer science is the discipline where linear algebra is used most heavily. Most real-life problems can be traced back to solving the linear equation of the form $Ax=b$, and computers are really good at it. However, the way that is used is generally the famous algorithm of Gaussian Elimination. However, based on linear algebra, scientists have developed sophisticated algorithms. Linear algebra is used to store different attributes of subjects in a concise manner and deal with them through the help of matrices. Machine learning, deep learning, data analysis, and characterization techniques such as SVD and PCA are eventually made from the use of linear algebra. These famous machine learning algorithms are being used as the building blocks of artificial intelligence deep learning.

\subsection{Quantum Mechanics}

When classical mechanics started to fail and quantum mechanics were being developed, Dirac and other people started the wave of matrix mechanics that can accurately describe quantum systems. Quantum Mechanics and a lot of related physics are based on linear algebra and matrices, diagonalization, eigenvalue finding, and several other applications are directly based on linear algebra and are used heavily to characterize literally any systems or materials out there. Even in classical field theory and high-energy physics disciplines such as string theory, particle physics heavily uses matrix mechanics.

\subsection{Data Visualization}

The modern era of data science is thriving on the basis of the analysis and handling of data in a seamless and convenient manner. Matrix mechanics and linear algebra are at the forefront of this journey. 


